% Options for packages loaded elsewhere
\PassOptionsToPackage{unicode}{hyperref}
\PassOptionsToPackage{hyphens}{url}
%
\documentclass[
]{article}
\title{Vaccination Mini Project}
\author{Ralph Goguanco}
\date{3/11/2022}

\usepackage{amsmath,amssymb}
\usepackage{lmodern}
\usepackage{iftex}
\ifPDFTeX
  \usepackage[T1]{fontenc}
  \usepackage[utf8]{inputenc}
  \usepackage{textcomp} % provide euro and other symbols
\else % if luatex or xetex
  \usepackage{unicode-math}
  \defaultfontfeatures{Scale=MatchLowercase}
  \defaultfontfeatures[\rmfamily]{Ligatures=TeX,Scale=1}
\fi
% Use upquote if available, for straight quotes in verbatim environments
\IfFileExists{upquote.sty}{\usepackage{upquote}}{}
\IfFileExists{microtype.sty}{% use microtype if available
  \usepackage[]{microtype}
  \UseMicrotypeSet[protrusion]{basicmath} % disable protrusion for tt fonts
}{}
\makeatletter
\@ifundefined{KOMAClassName}{% if non-KOMA class
  \IfFileExists{parskip.sty}{%
    \usepackage{parskip}
  }{% else
    \setlength{\parindent}{0pt}
    \setlength{\parskip}{6pt plus 2pt minus 1pt}}
}{% if KOMA class
  \KOMAoptions{parskip=half}}
\makeatother
\usepackage{xcolor}
\IfFileExists{xurl.sty}{\usepackage{xurl}}{} % add URL line breaks if available
\IfFileExists{bookmark.sty}{\usepackage{bookmark}}{\usepackage{hyperref}}
\hypersetup{
  pdftitle={Vaccination Mini Project},
  pdfauthor={Ralph Goguanco},
  hidelinks,
  pdfcreator={LaTeX via pandoc}}
\urlstyle{same} % disable monospaced font for URLs
\usepackage[margin=1in]{geometry}
\usepackage{color}
\usepackage{fancyvrb}
\newcommand{\VerbBar}{|}
\newcommand{\VERB}{\Verb[commandchars=\\\{\}]}
\DefineVerbatimEnvironment{Highlighting}{Verbatim}{commandchars=\\\{\}}
% Add ',fontsize=\small' for more characters per line
\usepackage{framed}
\definecolor{shadecolor}{RGB}{248,248,248}
\newenvironment{Shaded}{\begin{snugshade}}{\end{snugshade}}
\newcommand{\AlertTok}[1]{\textcolor[rgb]{0.94,0.16,0.16}{#1}}
\newcommand{\AnnotationTok}[1]{\textcolor[rgb]{0.56,0.35,0.01}{\textbf{\textit{#1}}}}
\newcommand{\AttributeTok}[1]{\textcolor[rgb]{0.77,0.63,0.00}{#1}}
\newcommand{\BaseNTok}[1]{\textcolor[rgb]{0.00,0.00,0.81}{#1}}
\newcommand{\BuiltInTok}[1]{#1}
\newcommand{\CharTok}[1]{\textcolor[rgb]{0.31,0.60,0.02}{#1}}
\newcommand{\CommentTok}[1]{\textcolor[rgb]{0.56,0.35,0.01}{\textit{#1}}}
\newcommand{\CommentVarTok}[1]{\textcolor[rgb]{0.56,0.35,0.01}{\textbf{\textit{#1}}}}
\newcommand{\ConstantTok}[1]{\textcolor[rgb]{0.00,0.00,0.00}{#1}}
\newcommand{\ControlFlowTok}[1]{\textcolor[rgb]{0.13,0.29,0.53}{\textbf{#1}}}
\newcommand{\DataTypeTok}[1]{\textcolor[rgb]{0.13,0.29,0.53}{#1}}
\newcommand{\DecValTok}[1]{\textcolor[rgb]{0.00,0.00,0.81}{#1}}
\newcommand{\DocumentationTok}[1]{\textcolor[rgb]{0.56,0.35,0.01}{\textbf{\textit{#1}}}}
\newcommand{\ErrorTok}[1]{\textcolor[rgb]{0.64,0.00,0.00}{\textbf{#1}}}
\newcommand{\ExtensionTok}[1]{#1}
\newcommand{\FloatTok}[1]{\textcolor[rgb]{0.00,0.00,0.81}{#1}}
\newcommand{\FunctionTok}[1]{\textcolor[rgb]{0.00,0.00,0.00}{#1}}
\newcommand{\ImportTok}[1]{#1}
\newcommand{\InformationTok}[1]{\textcolor[rgb]{0.56,0.35,0.01}{\textbf{\textit{#1}}}}
\newcommand{\KeywordTok}[1]{\textcolor[rgb]{0.13,0.29,0.53}{\textbf{#1}}}
\newcommand{\NormalTok}[1]{#1}
\newcommand{\OperatorTok}[1]{\textcolor[rgb]{0.81,0.36,0.00}{\textbf{#1}}}
\newcommand{\OtherTok}[1]{\textcolor[rgb]{0.56,0.35,0.01}{#1}}
\newcommand{\PreprocessorTok}[1]{\textcolor[rgb]{0.56,0.35,0.01}{\textit{#1}}}
\newcommand{\RegionMarkerTok}[1]{#1}
\newcommand{\SpecialCharTok}[1]{\textcolor[rgb]{0.00,0.00,0.00}{#1}}
\newcommand{\SpecialStringTok}[1]{\textcolor[rgb]{0.31,0.60,0.02}{#1}}
\newcommand{\StringTok}[1]{\textcolor[rgb]{0.31,0.60,0.02}{#1}}
\newcommand{\VariableTok}[1]{\textcolor[rgb]{0.00,0.00,0.00}{#1}}
\newcommand{\VerbatimStringTok}[1]{\textcolor[rgb]{0.31,0.60,0.02}{#1}}
\newcommand{\WarningTok}[1]{\textcolor[rgb]{0.56,0.35,0.01}{\textbf{\textit{#1}}}}
\usepackage{longtable,booktabs,array}
\usepackage{calc} % for calculating minipage widths
% Correct order of tables after \paragraph or \subparagraph
\usepackage{etoolbox}
\makeatletter
\patchcmd\longtable{\par}{\if@noskipsec\mbox{}\fi\par}{}{}
\makeatother
% Allow footnotes in longtable head/foot
\IfFileExists{footnotehyper.sty}{\usepackage{footnotehyper}}{\usepackage{footnote}}
\makesavenoteenv{longtable}
\usepackage{graphicx}
\makeatletter
\def\maxwidth{\ifdim\Gin@nat@width>\linewidth\linewidth\else\Gin@nat@width\fi}
\def\maxheight{\ifdim\Gin@nat@height>\textheight\textheight\else\Gin@nat@height\fi}
\makeatother
% Scale images if necessary, so that they will not overflow the page
% margins by default, and it is still possible to overwrite the defaults
% using explicit options in \includegraphics[width, height, ...]{}
\setkeys{Gin}{width=\maxwidth,height=\maxheight,keepaspectratio}
% Set default figure placement to htbp
\makeatletter
\def\fps@figure{htbp}
\makeatother
\setlength{\emergencystretch}{3em} % prevent overfull lines
\providecommand{\tightlist}{%
  \setlength{\itemsep}{0pt}\setlength{\parskip}{0pt}}
\setcounter{secnumdepth}{-\maxdimen} % remove section numbering
\ifLuaTeX
  \usepackage{selnolig}  % disable illegal ligatures
\fi

\begin{document}
\maketitle

The goal of this hands-on mini-project is to examine and compare the
Covid-19 vaccination rates around San Diego.

\begin{Shaded}
\begin{Highlighting}[]
\CommentTok{\# Import vaccination data}
\NormalTok{vax }\OtherTok{\textless{}{-}} \FunctionTok{read.csv}\NormalTok{(}\StringTok{"covid19vaccinesbyzipcode\_test.csv"}\NormalTok{)}
\FunctionTok{head}\NormalTok{(vax)}
\end{Highlighting}
\end{Shaded}

\begin{verbatim}
##   as_of_date zip_code_tabulation_area local_health_jurisdiction     county
## 1 2021-01-05                    95959                    Nevada     Nevada
## 2 2021-01-05                    95694                      Yolo       Yolo
## 3 2021-01-05                    95714                    Placer     Placer
## 4 2021-01-05                    95843                Sacramento Sacramento
## 5 2021-01-05                    95935                      Yuba       Yuba
## 6 2021-01-05                    95970                    Colusa     Colusa
##   vaccine_equity_metric_quartile                 vem_source
## 1                              3 Healthy Places Index Score
## 2                              3 Healthy Places Index Score
## 3                              2    CDPH-Derived ZCTA Score
## 4                              3 Healthy Places Index Score
## 5                              3    CDPH-Derived ZCTA Score
## 6                              2    CDPH-Derived ZCTA Score
##   age12_plus_population age5_plus_population persons_fully_vaccinated
## 1               16558.5                17604                       NA
## 2                8658.1                 9646                       NA
## 3                 477.7                  511                       NA
## 4               38637.1                44055                       NA
## 5                 206.0                  206                       NA
## 6                 564.9                  611                       NA
##   persons_partially_vaccinated percent_of_population_fully_vaccinated
## 1                           NA                                     NA
## 2                           NA                                     NA
## 3                           NA                                     NA
## 4                           NA                                     NA
## 5                           NA                                     NA
## 6                           NA                                     NA
##   percent_of_population_partially_vaccinated
## 1                                         NA
## 2                                         NA
## 3                                         NA
## 4                                         NA
## 5                                         NA
## 6                                         NA
##   percent_of_population_with_1_plus_dose booster_recip_count
## 1                                     NA                  NA
## 2                                     NA                  NA
## 3                                     NA                  NA
## 4                                     NA                  NA
## 5                                     NA                  NA
## 6                                     NA                  NA
##                                                                redacted
## 1 Information redacted in accordance with CA state privacy requirements
## 2 Information redacted in accordance with CA state privacy requirements
## 3 Information redacted in accordance with CA state privacy requirements
## 4 Information redacted in accordance with CA state privacy requirements
## 5 Information redacted in accordance with CA state privacy requirements
## 6 Information redacted in accordance with CA state privacy requirements
\end{verbatim}

\begin{quote}
Q1. What column details the total number of people fully vaccinated?
\end{quote}

persons\_fully\_vaccinated

\begin{quote}
Q2. What column details the Zip code tabulation area?
\end{quote}

zip\_code\_tabulation\_area

\begin{quote}
Q3. What is the earliest date in this dataset?
\end{quote}

\begin{Shaded}
\begin{Highlighting}[]
\FunctionTok{head}\NormalTok{(vax}\SpecialCharTok{$}\NormalTok{as\_of\_date)}
\end{Highlighting}
\end{Shaded}

\begin{verbatim}
## [1] "2021-01-05" "2021-01-05" "2021-01-05" "2021-01-05" "2021-01-05"
## [6] "2021-01-05"
\end{verbatim}

\begin{Shaded}
\begin{Highlighting}[]
\CommentTok{\#2021{-}01{-}05}
\end{Highlighting}
\end{Shaded}

\begin{quote}
Q4. What is the latest date in this dataset?
\end{quote}

\begin{Shaded}
\begin{Highlighting}[]
\FunctionTok{tail}\NormalTok{(vax}\SpecialCharTok{$}\NormalTok{as\_of\_date)}
\end{Highlighting}
\end{Shaded}

\begin{verbatim}
## [1] "2022-03-08" "2022-03-08" "2022-03-08" "2022-03-08" "2022-03-08"
## [6] "2022-03-08"
\end{verbatim}

\begin{Shaded}
\begin{Highlighting}[]
\CommentTok{\#2022{-}03{-}08}
\end{Highlighting}
\end{Shaded}

As we have done previously, let's call the skim() function from the
skimr package to get a quick overview of this dataset:

\begin{Shaded}
\begin{Highlighting}[]
\NormalTok{skimr}\SpecialCharTok{::}\FunctionTok{skim}\NormalTok{(vax)}
\end{Highlighting}
\end{Shaded}

\begin{longtable}[]{@{}ll@{}}
\caption{Data summary}\tabularnewline
\toprule
\endhead
Name & vax \\
Number of rows & 109368 \\
Number of columns & 15 \\
\_\_\_\_\_\_\_\_\_\_\_\_\_\_\_\_\_\_\_\_\_\_\_ & \\
Column type frequency: & \\
character & 5 \\
numeric & 10 \\
\_\_\_\_\_\_\_\_\_\_\_\_\_\_\_\_\_\_\_\_\_\_\_\_ & \\
Group variables & None \\
\bottomrule
\end{longtable}

\textbf{Variable type: character}

\begin{longtable}[]{@{}lrrrrrrr@{}}
\toprule
skim\_variable & n\_missing & complete\_rate & min & max & empty &
n\_unique & whitespace \\
\midrule
\endhead
as\_of\_date & 0 & 1 & 10 & 10 & 0 & 62 & 0 \\
local\_health\_jurisdiction & 0 & 1 & 0 & 15 & 310 & 62 & 0 \\
county & 0 & 1 & 0 & 15 & 310 & 59 & 0 \\
vem\_source & 0 & 1 & 15 & 26 & 0 & 3 & 0 \\
redacted & 0 & 1 & 2 & 69 & 0 & 2 & 0 \\
\bottomrule
\end{longtable}

\textbf{Variable type: numeric}

\begin{longtable}[]{@{}
  >{\raggedright\arraybackslash}p{(\columnwidth - 20\tabcolsep) * \real{0.26}}
  >{\raggedleft\arraybackslash}p{(\columnwidth - 20\tabcolsep) * \real{0.06}}
  >{\raggedleft\arraybackslash}p{(\columnwidth - 20\tabcolsep) * \real{0.08}}
  >{\raggedleft\arraybackslash}p{(\columnwidth - 20\tabcolsep) * \real{0.05}}
  >{\raggedleft\arraybackslash}p{(\columnwidth - 20\tabcolsep) * \real{0.05}}
  >{\raggedleft\arraybackslash}p{(\columnwidth - 20\tabcolsep) * \real{0.04}}
  >{\raggedleft\arraybackslash}p{(\columnwidth - 20\tabcolsep) * \real{0.05}}
  >{\raggedleft\arraybackslash}p{(\columnwidth - 20\tabcolsep) * \real{0.05}}
  >{\raggedleft\arraybackslash}p{(\columnwidth - 20\tabcolsep) * \real{0.05}}
  >{\raggedleft\arraybackslash}p{(\columnwidth - 20\tabcolsep) * \real{0.05}}
  >{\raggedright\arraybackslash}p{(\columnwidth - 20\tabcolsep) * \real{0.24}}@{}}
\toprule
\begin{minipage}[b]{\linewidth}\raggedright
skim\_variable
\end{minipage} & \begin{minipage}[b]{\linewidth}\raggedleft
n\_missing
\end{minipage} & \begin{minipage}[b]{\linewidth}\raggedleft
complete\_rate
\end{minipage} & \begin{minipage}[b]{\linewidth}\raggedleft
mean
\end{minipage} & \begin{minipage}[b]{\linewidth}\raggedleft
sd
\end{minipage} & \begin{minipage}[b]{\linewidth}\raggedleft
p0
\end{minipage} & \begin{minipage}[b]{\linewidth}\raggedleft
p25
\end{minipage} & \begin{minipage}[b]{\linewidth}\raggedleft
p50
\end{minipage} & \begin{minipage}[b]{\linewidth}\raggedleft
p75
\end{minipage} & \begin{minipage}[b]{\linewidth}\raggedleft
p100
\end{minipage} & \begin{minipage}[b]{\linewidth}\raggedright
hist
\end{minipage} \\
\midrule
\endhead
zip\_code\_tabulation\_area & 0 & 1.00 & 93665.11 & 1817.39 & 90001 &
92257.75 & 93658.50 & 95380.50 & 97635.0 & ▃▅▅▇▁ \\
vaccine\_equity\_metric\_quartile & 5394 & 0.95 & 2.44 & 1.11 & 1 & 1.00
& 2.00 & 3.00 & 4.0 & ▇▇▁▇▇ \\
age12\_plus\_population & 0 & 1.00 & 18895.04 & 18993.91 & 0 & 1346.95 &
13685.10 & 31756.12 & 88556.7 & ▇▃▂▁▁ \\
age5\_plus\_population & 0 & 1.00 & 20875.24 & 21106.01 & 0 & 1460.50 &
15364.00 & 34877.00 & 101902.0 & ▇▃▂▁▁ \\
persons\_fully\_vaccinated & 18494 & 0.83 & 12246.65 & 13155.33 & 11 &
1074.00 & 7453.00 & 20138.00 & 79763.0 & ▇▃▁▁▁ \\
persons\_partially\_vaccinated & 18494 & 0.83 & 848.69 & 1401.05 & 11 &
77.00 & 377.00 & 1091.00 & 36844.0 & ▇▁▁▁▁ \\
percent\_of\_population\_fully\_vaccinated & 18494 & 0.83 & 0.51 & 0.26
& 0 & 0.34 & 0.55 & 0.71 & 1.0 & ▅▅▇▇▃ \\
percent\_of\_population\_partially\_vaccinated & 18494 & 0.83 & 0.05 &
0.10 & 0 & 0.02 & 0.03 & 0.05 & 1.0 & ▇▁▁▁▁ \\
percent\_of\_population\_with\_1\_plus\_dose & 18494 & 0.83 & 0.55 &
0.28 & 0 & 0.36 & 0.59 & 0.76 & 1.0 & ▅▃▆▇▅ \\
booster\_recip\_count & 64441 & 0.41 & 4265.63 & 6073.54 & 11 & 183.00 &
1207.00 & 6484.00 & 51001.0 & ▇▁▁▁▁ \\
\bottomrule
\end{longtable}

\begin{quote}
Q5. How many numeric columns are in this dataset?
\end{quote}

9 \textgreater Q6. Note that there are ``missing values'' in the
dataset. How many NA values there in the persons\_fully\_vaccinated
column?

\begin{Shaded}
\begin{Highlighting}[]
\NormalTok{N }\OtherTok{\textless{}{-}} \FunctionTok{sum}\NormalTok{(}\FunctionTok{is.na}\NormalTok{(vax}\SpecialCharTok{$}\NormalTok{persons\_fully\_vaccinated))}
\NormalTok{total }\OtherTok{\textless{}{-}} \FunctionTok{sum}\NormalTok{(vax}\SpecialCharTok{$}\NormalTok{percent\_of\_population\_fully\_vaccinated, }\AttributeTok{na.rm =} \ConstantTok{TRUE}\NormalTok{)}
\end{Highlighting}
\end{Shaded}

\begin{quote}
What percent of persons\_fully\_vaccinated values are missing (to 2
significant figures)?
\end{quote}

\begin{Shaded}
\begin{Highlighting}[]
\NormalTok{N}\SpecialCharTok{/}\NormalTok{total}
\end{Highlighting}
\end{Shaded}

\begin{verbatim}
## [1] 0.3959551
\end{verbatim}

\#Working with dates

\begin{Shaded}
\begin{Highlighting}[]
\CommentTok{\#today()}

\CommentTok{\# Specify that we are using the year{-}month{-}day format}
\CommentTok{\#vax$as\_of\_date \textless{}{-} ymd(vax$as\_of\_date)}
\end{Highlighting}
\end{Shaded}

\begin{Shaded}
\begin{Highlighting}[]
\CommentTok{\#today() {-} vax$as\_of\_date[1]}
\end{Highlighting}
\end{Shaded}

\begin{quote}
Q9. How many days have passed since the last update of the dataset? 3
days
\end{quote}

\begin{Shaded}
\begin{Highlighting}[]
\CommentTok{\#today() {-} vax$as\_of\_date[109358]}
\end{Highlighting}
\end{Shaded}

\begin{quote}
Q10. How many unique dates are in the dataset (i.e.~how many different
dates are detailed)? 62
\end{quote}

\begin{Shaded}
\begin{Highlighting}[]
\NormalTok{date\_uniq }\OtherTok{\textless{}{-}} \FunctionTok{unique}\NormalTok{(vax}\SpecialCharTok{$}\NormalTok{as\_of\_date)}
\FunctionTok{length}\NormalTok{(date\_uniq)}
\end{Highlighting}
\end{Shaded}

\begin{verbatim}
## [1] 62
\end{verbatim}

\#Working with ZIP codes

\begin{Shaded}
\begin{Highlighting}[]
\CommentTok{\# Pull data for all ZIP codes in the dataset}
\CommentTok{\#zipdata \textless{}{-} reverse\_zipcode( vax$zip\_code\_tabulation\_area )}
\end{Highlighting}
\end{Shaded}

\begin{Shaded}
\begin{Highlighting}[]
\CommentTok{\# Subset to San Diego county only areas}
\NormalTok{sd }\OtherTok{\textless{}{-}}\NormalTok{ vax[ }\DecValTok{4}\NormalTok{ , ]}
\end{Highlighting}
\end{Shaded}

\begin{Shaded}
\begin{Highlighting}[]
\CommentTok{\# \textless{}{-} filter(vax, county == "San Diego")}

\CommentTok{\#nrow(sd)}
\end{Highlighting}
\end{Shaded}

\begin{quote}
Q11. How many distinct zip codes are listed for San Diego County?
\end{quote}

\begin{Shaded}
\begin{Highlighting}[]
\FunctionTok{length}\NormalTok{(}\FunctionTok{unique}\NormalTok{(sd))}
\end{Highlighting}
\end{Shaded}

\begin{verbatim}
## [1] 15
\end{verbatim}

\begin{quote}
Q12. What San Diego County Zip code area has the largest 12 + Population
in this dataset?
\end{quote}

\begin{Shaded}
\begin{Highlighting}[]
\FunctionTok{which.max}\NormalTok{(sd[, }\DecValTok{7}\NormalTok{])}
\end{Highlighting}
\end{Shaded}

\begin{verbatim}
## [1] 1
\end{verbatim}

\begin{Shaded}
\begin{Highlighting}[]
\CommentTok{\# Zip Code in \#103 is 92154}
\end{Highlighting}
\end{Shaded}

\begin{Shaded}
\begin{Highlighting}[]
\CommentTok{\#sd1 \textless{}{-} filter(sd, as\_of\_date == "2022{-}02{-}22")}
\end{Highlighting}
\end{Shaded}

\begin{quote}
Q13. What is the overall average ``Percent of Population Fully
Vaccinated'' value for all San Diego ``County'' as of ``2022-02-22''?
\end{quote}

\begin{Shaded}
\begin{Highlighting}[]
\CommentTok{\#mean(sd1$percent\_of\_population\_fully\_vaccinated, na.rm = TRUE)}
\end{Highlighting}
\end{Shaded}

71.46\%

\begin{quote}
Q14. Using either ggplot or base R graphics make a summary figure that
shows the distribution of Percent of Population Fully Vaccinated values
as of ``2022-02-22''?
\end{quote}

\begin{Shaded}
\begin{Highlighting}[]
\CommentTok{\#hist(sd$percent\_of\_population\_fully\_vaccinated)}
\end{Highlighting}
\end{Shaded}

Focus on UCSD/La Jolla

\begin{Shaded}
\begin{Highlighting}[]
\CommentTok{\#ucsd \textless{}{-} filter(sd, zip\_code\_tabulation\_area=="92037")}
\CommentTok{\#ucsd[1,]$age5\_plus\_population}
\end{Highlighting}
\end{Shaded}

\begin{quote}
Q15. Using ggplot make a graph of the vaccination rate time course for
the 92037 ZIP code area:
\end{quote}

\begin{Shaded}
\begin{Highlighting}[]
\CommentTok{\#ggplot(ucsd) +}
\CommentTok{\#  aes(x=ucsd$as\_of\_date,}
\CommentTok{\#      y = ucsd$percent\_of\_population\_fully\_vaccinated) +}
\CommentTok{\#  geom\_point() +}
\CommentTok{\#  geom\_line(group=1) +}
\CommentTok{\#  ylim(c(0,1)) +}
\CommentTok{\#  labs(x="Date", y="Percent Vaccinated")}
\end{Highlighting}
\end{Shaded}

Comparing to similar sized areas

\begin{Shaded}
\begin{Highlighting}[]
\CommentTok{\# Subset to all CA areas with a population as large as 92037}
\CommentTok{\#vax.36 \textless{}{-} filter(vax, age5\_plus\_population \textgreater{} 36144 \&}
\CommentTok{\#                as\_of\_date == "2022{-}02{-}22")}

\CommentTok{\#head(vax.36)}
\end{Highlighting}
\end{Shaded}

\begin{quote}
Q16. Calculate the mean ``Percent of Population Fully Vaccinated'' for
ZIP code areas with a population as large as 92037 (La Jolla)
as\_of\_date ``2022-02-22''. Add this as a straight horizontal line to
your plot from above with the geom\_hline() function?
\end{quote}

\begin{Shaded}
\begin{Highlighting}[]
\CommentTok{\#ggplot(ucsd) +}
\CommentTok{\#aes(x=ucsd$as\_of\_date,}
\CommentTok{\#      y = ucsd$percent\_of\_population\_fully\_vaccinated) +}
\CommentTok{\#  geom\_point() +}
\CommentTok{\#  geom\_line(group=1) +}
\CommentTok{\#  ylim(c(0,1)) +}
\CommentTok{\#  labs(x="Date", y="Percent Vaccinated") +}
\CommentTok{\#  geom\_hline(yintercept = 0.732736)}
\end{Highlighting}
\end{Shaded}

\begin{quote}
Q17. What is the 6 number summary (Min, 1st Qu., Median, Mean, 3rd Qu.,
and Max) of the ``Percent of Population Fully Vaccinated'' values for
ZIP code areas with a population as large as 92037 (La Jolla)
as\_of\_date ``2022-02-22''?
\end{quote}

\begin{Shaded}
\begin{Highlighting}[]
\CommentTok{\#fivenum(vax.36)}
\end{Highlighting}
\end{Shaded}

\begin{quote}
Q18. Using ggplot generate a histogram of this data.
\end{quote}

\begin{Shaded}
\begin{Highlighting}[]
\CommentTok{\#ggplot(vax.36, aes(x=vax.36$percent\_of\_population\_fully\_vaccinated)) + }
\CommentTok{\#geom\_histogram()}
\end{Highlighting}
\end{Shaded}

\begin{quote}
Q19. Is the 92109 and 92040 ZIP code areas above or below the average
value you calculated for all these above?
\end{quote}

\begin{Shaded}
\begin{Highlighting}[]
\CommentTok{\#vax \%\textgreater{}\% filter(as\_of\_date == "2022{-}02{-}22") \%\textgreater{}\%  }
\CommentTok{\#  filter(zip\_code\_tabulation\_area=="92040") \%\textgreater{}\%}
\CommentTok{\#  select(percent\_of\_population\_fully\_vaccinated)}
\end{Highlighting}
\end{Shaded}

Below the average value

\begin{quote}
Q20. Finally make a time course plot of vaccination progress for all
areas in the full dataset with a age5\_plus\_population \textgreater{}
36144.
\end{quote}

\begin{Shaded}
\begin{Highlighting}[]
\CommentTok{\#vax.36.all \textless{}{-} filter(vax, age5\_plus\_population \textgreater{} 36144)}


\CommentTok{\#ggplot(vax.36.all) +}
\CommentTok{\# aes(x = vax.36$as\_of\_date,}
\CommentTok{\#      percent\_of\_population\_fully\_vaccinated, }
\CommentTok{\#      group=zip\_code\_tabulation\_area) +}
\CommentTok{\#  geom\_line(alpha=0.2, color=vax) +}
\CommentTok{\#  ylim(vax.36) +}
\CommentTok{\# labs(x="Date" , y="Percent",}
\CommentTok{\#       title="Vaccination rate accross California",}
\CommentTok{\#       subtitle="Only areas with a population above 36k are shown") +}
\CommentTok{\#  geom\_hline(yintercept = 0.7327, linetype=dash)}
\end{Highlighting}
\end{Shaded}

\begin{quote}
Q21. How do you feel about traveling for Spring Break and meeting for
in-person class afterwards?
\end{quote}

I feel better knowing that the percent vaccinated will only go up

\end{document}
